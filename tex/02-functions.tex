\documentclass[english,serif,mathserif,xcolor=pdftex,dvipsnames,table]{beamer}

\usepackage[T1]{fontenc}
\usepackage[utf8x]{inputenc}

\usetheme[informal]{s3it}
\usepackage{s3it}


\title[2. Functions]{%
  A Short and Incomplete Introduction to Julia
}
\subtitle{\bfseries Part 2: Functions}
\author[R.~Murri]{%
  \textbf{Riccardo Murri} \texttt{<riccardo.murri@uzh.ch>}
  \\
  S3IT: Services and Support for Science IT,
  \\
  University of Zurich
}
\date{September 12, 2019}


%% simulate IJulia prompt
\newcounter{prompt}
\newcommand\I{%
  \global\def\promptnr{\arabic{prompt}}%
  \stepcounter{prompt}%
  {\color{blue}\bfseries In~[\promptnr]:}%
}
\renewcommand\O{{\color{red}\bfseries Out[\promptnr]:}}
\newcommand\C{{\color{blue}   ...:}}

\begin{document}

% title frame
\maketitle


\part{Functions}

\begin{frame}[fragile,label=func1]
  \frametitle{Functions, I}
  Functions are called by postfixing the function name with a
  parenthesized argument list.

  \+
\begin{semiverbatim}
\julia string(42)
"42"
\julia string("x=", 42)
"x=42"
\julia log10(1e42)
42.0
\julia √(1764)  # {\textbackslash}sqrt + TAB
42.0
\end{semiverbatim}
\end{frame}


\begin{frame}[fragile,fragile]
  \frametitle{Functions, II}
  Some functions can take a variable number of arguments. For instance:

  \+
  \begin{description}
    \item[max($x_0$, \ldots, $x_n$)] Return the maximum of $\{ x_0, \ldots, x_n \}$
    \item[min($x_0$, \ldots, $x_n$)] Return the minimum of $\{ x_0, \ldots, x_n \}$
    \item[string($x_0$, \ldots, $x_n$)] Return the concatenation of printed representation of all arguments.
  \end{description}

  \+
  Examples:
\begin{semiverbatim}
\julia max(1,2,3)
3
\julia string("x=",1,2,3)
"x=123"
\end{semiverbatim}
\end{frame}


\begin{frame}[fragile]
  \frametitle{How to define new functions}
  \begin{columns}[t]
    \begin{column}{0.5\textwidth}
\begin{lstlisting}
"""
A friendly function.
"""
~\HL{\textbf{function}}~ greet(name)
  println("Hello, $(name)!)
~\HL{\textbf{end}}~

# the customary greeting
greet("world")
\end{lstlisting}
    \end{column}
    \begin{column}{0.5\textwidth}
      \raggedleft
      A function definition is enclosed by the keyword \textbf{function} and a matching \textbf{end}.
    \end{column}
  \end{columns}
\end{frame}


\begin{frame}[fragile]
  \begin{columns}[t]
    \begin{column}{0.5\textwidth}
\begin{lstlisting}
"""
A friendly function.
"""
function greet(name)
  println("Hello, $(name)!)
end

~\HL{\it\tt\#\ the customary greeting}~
greet("world")
\end{lstlisting}
    \end{column}
    \begin{column}{0.5\textwidth}
      \raggedleft
       (This is a comment. It is ignored by Julia, just like blank lines.)
    \end{column}
  \end{columns}
\end{frame}


\begin{frame}[fragile]
  \begin{columns}[t]
    \begin{column}{0.5\textwidth}
\begin{lstlisting}
"""
A friendly function.
"""
function greet(name)
  println("Hello, $(name)!)
end

# the customary greeting
~\HL{greet("world")}~
\end{lstlisting}
    \end{column}
    \begin{column}{0.5\textwidth}
      \raggedleft
      This calls the function \\ just defined.
    \end{column}
  \end{columns}
\end{frame}


\begin{frame}[fragile]
  \begin{columns}[t]
    \begin{column}{0.5\textwidth}
\begin{lstlisting}
~\HL{"""}~
~\HL{A friendly function.}~
~\HL{"""}~
function greet(name)
  println("Hello, $(name)!)
end

# the customary greeting
greet("world")
\end{lstlisting}
    \end{column}
    \begin{column}{0.5\textwidth}
      \raggedleft
      What is this? The answer in the next exercise!
    \end{column}
  \end{columns}
\end{frame}


\begin{frame}
  \begin{exercise*}[2.A]
    Type and run the code on the previous page at the interactive
    prompt.

    \+
    What's the result of evaluating the function \texttt{greet("world")}?

    \+
    What does \texttt{?greet} output?
  \end{exercise*}
\end{frame}

\begin{frame}[fragile]
  \frametitle{Default values}
  \begin{columns}[t]
    \begin{column}{0.5\textwidth}
\begin{lstlisting}
~\julia~
  function greet(~\HL{name="world"}~)
    println("Hello, $(name)!")
  end
~\hbox{greet (generic function with 2 methods)}~

~\julia~ greet()
Hello, world!
\end{lstlisting}
    \end{column}
    \begin{column}{0.5\textwidth}
      \raggedleft
      Function arguments can have default values.
    \end{column}
  \end{columns}
\end{frame}

% \begin{frame}[fragile]
%   \frametitle{How to define new functions}
%   \begin{columns}[t]
%     \begin{column}{0.5\textwidth}
% \begin{lstlisting}
% """
% A friendly function.
% """
% function greet(name)
%   println("Hello, $(name)!)
% end

% # the customary greeting
% greet("world")
% \end{lstlisting}
%     \end{column}
%     \begin{column}{0.5\textwidth}
%       \raggedleft

%     \end{column}
%   \end{columns}
% \end{frame}


% \begin{frame}[fragile,fragile,fragile]
%   \frametitle{Named arguments}
% You can mark some function arguments as being passed ``by name only'':
% \begin{semiverbatim}
% \julia function geomseq(n, r\HL{;} a=1)
%        collect(a*r^k for k in 1:n)
%        end
% geomseq (generic function with 1 method)
% \end{semiverbatim}

% \+ All arguments coming after the `\texttt{;}' character must be given
% using the \texttt{name=value} syntax when calling the function:
% \begin{semiverbatim}
% \julia geomseq(2, 4, a=3)
% 2-element Array{Int64,1}:
%  12
%  48
% \julia geomseq(2, 4, 3)
% \ERROR{MethodError: no method matching geomseq(::Int64, ::Int64, ::Int64)}
% \end{semiverbatim}
% \end{frame}


\begin{frame}[fragile]
  \frametitle{The return value of functions}
  A function call returns the value of the expression evaluated last:
\begin{semiverbatim}
\julia function f()
       1
       2
       3
       end
f (generic function with 1 method)

\julia f()
3
\end{semiverbatim}
\end{frame}

\begin{frame}[fragile]
  \frametitle{The \texttt{return} keyword}

  The {\ttfamily\bfseries return} keyword ends execution of a function.

  \+ The expression following `\texttt{return}' sets the return value
  of the function being called.

  \+
\begin{semiverbatim}
\julia function f()
       println("entering f()...")
       return 2
       println("f() is done by now")
       end
f (generic function with 1 method)

\julia f()
entering f()...
2
\end{semiverbatim}
\end{frame}


\begin{frame}[fragile]
  \frametitle{The special value `nothing'}
  If a \texttt{return} keyword is \emph{not} followed by any
  expression, then the function returns a special value called
  `\texttt{nothing}':
\begin{semiverbatim}
\julia \lstinline|function noval()|
       \lstinline|return;  # no expression to evaluate|
       \lstinline|end|
noval (generic function with 1 method)
\end{semiverbatim}

  \+ The constant `\texttt{nothing}' has its own special type `\texttt{Nothing}':
\begin{semiverbatim}
\julia \lstinline|typeof(noval())|
Nothing
\end{semiverbatim}
\end{frame}


\part{Basic control flow}
\begin{frame}[fragile]
  \frametitle{Conditionals}
  Conditional execution uses the \texttt{if} statement:
\begin{lstlisting}
if ~\it test-expr~
  # evaluate these expressions
elseif ~\it other-test-expr~
  # evaluate these
else
  # executed if none of the above matched
end
\end{lstlisting}

  \+The \texttt{elseif} can be repeated, with different conditions, or
  left out entirely.

  \+
  Also the \texttt{else} clause is optional.
\end{frame}


\begin{frame}[fragile]
  \frametitle{Truth values, I}
  \small

  Test expressions used in \texttt{if} and \texttt{elseif} clauses
  \emph{must} evaluate to a boolean value: one of the two constants
  `\texttt{true}' or `\texttt{false}'.

  \+
  All relational and comparison operators return boolean values:
  \begin{columns}
    \begin{column}{0.5\textwidth}
\begin{semiverbatim}
\julia \textbf{if} 1 > 0
    println("A truism.")
  \textbf{else}
    print("It's false!!")
  \textbf{end}
A truism.
\end{semiverbatim}
    \end{column}
    \begin{column}{0.5\textwidth}
\begin{semiverbatim}
\julia \textbf{if} NaN == NaN
    println("A truism.")
  \textbf{else}
    print("It's false!!")
  \textbf{end}
It's false!
\end{semiverbatim}
    \end{column}
  \end{columns}
\end{frame}


\begin{frame}[fragile]
  \frametitle{Truth values, II}
  \small

  Test expressions used in \texttt{if} and \texttt{elseif} clauses
  \emph{must} evaluate to a boolean value: one of the two constants
  `\texttt{true}' or `\texttt{false}'.

  \+ Integers, strings, or other types cannot be used \\ in \texttt{if}'s
  test expression:
\begin{semiverbatim}
\julia \textbf{if} 1  # Python would allow this!
    println("A truism.")
  \textbf{else}
    print("It's false!!")
  \textbf{end}
\ERROR{TypeError: non-boolean (Int64) used
in boolean context}
\end{semiverbatim}
\end{frame}


\begin{frame}[fragile]
  \frametitle{The ternary operator}
  All code is an expression in Julia, so an `\texttt{if}' block
  returns the value of the last expression evaluated:
\begin{lstlisting}
~\julia~ x = if (name == "Julia"); '~$\heartsuit$~' else '~$\spadesuit$~' end
'~$\spadesuit$~': Unicode U+2660 (category So: Symbol, other)
\end{lstlisting}

  \+ Testing and selecting one of two values is so common, it has an
  abbreviation:
\begin{lstlisting}
~\julia~ x = (name == "Julia") ~\textbf{?}~ '~$\heartsuit$~' ~\textbf{:}~ '~$\spadesuit$~'
'~$\spadesuit$~': Unicode U+2660 (category So: Symbol, other)
\end{lstlisting}
  Note that `\texttt{?}' must be preceded by a space.
\end{frame}


\begin{frame}[fragile]
  \frametitle{\texttt{while}-loops}
  Conditional looping uses the \texttt{while} statement:
\begin{lstlisting}
while ~\it test-expr~
  # expressions
end
\end{lstlisting}

  \+
  To break out of a \texttt{while} loop, use the \texttt{break}
  statement.

  \+
  Use the \texttt{continue} statement anywhere in the expression
  block to jump back to the \texttt{while} statement.

  \+ Again, all code is an expression in Julia, so also a
  `\texttt{while}' block returns the value of the last expression
  evaluated.
\end{frame}


\begin{frame}
  \begin{exercise*}[2.B]
    Modify the \texttt{greet()} function to print `$\heartsuit$'{}\footnote{Use
      \texttt{{\textbackslash}heartsuit} followed by the TAB key.} if
    the argument \texttt{name} is the string \texttt{'Julia'}.
  \end{exercise*}

  \+
  \begin{exercise*}[2.C]
    Write a function \texttt{greetmany()} that repeatedly:
    \begin{enumerate}
    \item     Asks the user for a name;
    \item     Prints a greeting to the person by that name.
    \end{enumerate}

    The function should keep asking until the user enters the single
    letter `\texttt{q}' as the name.

    Julia's built-in function \texttt{readline()} can be used to input a string.
  \end{exercise*}
\end{frame}


\part{Packages}

% \begin{frame}
%   \frametitle{Using functions defined in another file}
%   Julia code is organized into modules, files, and packages:

%   \begin{describe}{modules}
%     Modules are collections of variables and functions. A module is a
%     \emph{namespace}: bindings with the same name can coexist in the
%     different modules without interfering.
%   \end{describe}

%   \begin{describe}{files}
%     Files containing Julia code have the suffix `\texttt{.jl}'; they
%     must use the UTF-8 encoding.
%   \end{describe}

%   \begin{describe}{packages}
%     A package is a collection of files, structured in a way that
%     Julia's built-in package manager can recognize.
%   \end{describe}

%   \+
%   \begin{seealso}
%     \url{https://en.wikibooks.org/wiki/Introducing_Julia/Modules_and_packages}
%   \end{seealso}
% \end{frame}


\begin{frame}[fragile]
  \frametitle{Using functions defined in another file, I}
  \small

  Julia code is organized into modules, files, and packages.

  \+ You can make functions from other packages usable from the
  current module with the \texttt{using} keyword:
\begin{lstlisting}
~\julia~ @sprintf("working?")
~\ERROR{LoadError: UndefVarError: @sprintf not defined}~

~\julia~ ~\HL{\textbf{using} Printf}~

~\julia~ @sprintf("working!")
"working!"
\end{lstlisting}
\end{frame}

\begin{frame}[fragile,fragile]
  \frametitle{Using functions defined in another file, II}
  Using a package will \emph{not} overwrite functions:
\begin{lstlisting}
~\julia~ function red(c)
       "Defined in Main"
       end
~{red (generic function with 1 method)}~

~\julia~ using Colors

~\julia~ red == Colors.red
false

~\julia~ green == Colors.green
true
\end{lstlisting}
\end{frame}


\begin{frame}[fragile,fragile,fragile,fragile]
  \frametitle{Using functions defined in another file, III}
  \small
  To know in what module a function is defined, use the macro \texttt{@which}:
\begin{lstlisting}
~\julia~ @which red
Main

~\julia~ @which green
ColorTypes
\end{lstlisting}
  (\texttt{Main} is the module where Julia loads definitions by default.)

  \+
  To list all symbols (functions and variables) defined in a
  module, use the function \texttt{names(ModuleName)}:
\begin{lstlisting}
~\julia~ names(Printf)
3-element Array{Symbol,1}:
 Symbol("@printf")
 Symbol("@sprintf")
 :Printf
\end{lstlisting}
\end{frame}


\begin{frame}[fragile]
  \frametitle{Installing packages}
  The standard package \texttt{Pkg} can be used to install new packages:
\begin{lstlisting}
~\julia~ ~\HL{\textbf{using} Pkg}~

~\julia~ ~\HL{Pkg.add("Colors")}~
  Updating registry at `~\textasciitilde~/.julia/registries/General`
  ~\ldots~
 [no changes]
\end{lstlisting}

  Note that \texttt{\texttt{Pkg}} does not export its functions, so to
  call them one has to prefix the name with `\,\texttt{Pkg.}'
\end{frame}


\begin{frame}[fragile]
  \small

  For a better understanding of Julia's package system \\ and its (many) nuances, see:
  \begin{itemize}
  \+\item \url{https://en.wikibooks.org/wiki/Introducing_Julia/Modules_and_packages}
    Wikibook ``Introducing Julia'' for a practical summary of all the options.
  \+\item \url{https://docs.julialang.org/en/v1/manual/modules/} Julia's manual section on ``Modules'' for details of the relationship between modules, files, and packages.
  \+\item \url{https://docs.julialang.org/en/v1/manual/code-loading/}
    Julia's manual section on ``Code loading'' for details on how
    files are organized in packages and how to define local packages.
  \end{itemize}
\end{frame}


\end{document}

%%% Local Variables:
%%% mode: latex
%%% TeX-master: t
%%% End:
