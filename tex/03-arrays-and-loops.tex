\documentclass[english,serif,mathserif,xcolor=pdftex,dvipsnames,table]{beamer}

\usepackage[T1]{fontenc}
\usepackage[utf8x]{inputenc}

\usetheme[informal]{s3it}
\usepackage{s3it}

\title[3. Arrays]{%
  A Short and Incomplete Introduction to Julia
}
\subtitle{\bfseries Part 3: Vectors and Loops}
\author[R.~Murri]{%
  \textbf{Riccardo Murri} \texttt{<riccardo.murri@uzh.ch>}
  \\
  S3IT: Services and Support for Science IT,
  \\
  University of Zurich
}
\date{September 12, 2019}


\begin{document}

% title frame
\maketitle

\part{Array basics}
\begin{frame}[fragile,fragile]
  \frametitle{Constructing Vectors}

  Vectors (1D arrays) are created and initialized by enclosing values
  in square brackets `\texttt{[}' and `\texttt{]}':
\begin{lstlisting}
~\julia~ v = [1, 2, 3, 4, 5]
5-element Array{Int64,1}:
 1
 2
 3
 4
 5
\end{lstlisting}

  \+
  An empty array is created by a pair of square brackets with
  nothing in between:
\begin{lstlisting}
~\julia~ z = []
0-element Array{Any,1}
\end{lstlisting}
\end{frame}


\begin{frame}[fragile]
  \frametitle{The Type of an Array}
  Every value in Julia has a type \\ --- arrays are no exception:
\begin{semiverbatim}
\julia v = [1, 3, 3, 7];
\julia typeof(v)
\only<1>{\HL{Array\{Int64,1\}}}\only<2>{Array\{\HL{Int64},1\}}\only<3>{Array\{Int64,\HL{1}\}}
\end{semiverbatim}

  \+
  \texttt{\alt<1>{\HL{Array\{\ldots\}}}{Array\{\ldots\}}} indicates a \emph{parametric} type:
  \begin{itemize}
  \item \texttt{\alt<2>{\HL{Int64}}{Int64}} is the type of elements in the array \\ (also available as \texttt{eltype(a)});
  \item \texttt{\alt<3>{\HL{1}}{1}} is the number of dimensions.
  \end{itemize}
\end{frame}


\begin{frame}[fragile,fragile]
  \frametitle{Vectors, Matrices, Arrays}
  \footnotesize%\smaller

  Julia provides syntax for entering literal 1D and 2D arrays; \\
  note the usage of `\texttt{\HL{,}}' vs.~`\texttt{\HL{\space}}'
  vs.~`\texttt{\HL{;}}'

  \+
  \begin{tabular}[t]{p{0.45\linewidth}p{0.45\linewidth}}
    \begin{minipage}{1.0\linewidth}
      \raggedright
      1D arrays are entered with commas and represented as (column) vectors:
      \vspace{-1ex}
\begin{semiverbatim}
\julia v = [1\HL{,} 2\HL{,} 3]
3-element Array{Int64,1}:
 1
 2
 3
\end{semiverbatim}
    \end{minipage}
    &
    \begin{minipage}{1.0\linewidth}
      \vspace{-5ex}
      \raggedright
      Use space instead of comma for a ``wide'' matrix, which is
      represented as a row vector:
      \vspace{-1ex}
\begin{semiverbatim}
\julia w = [1\HL{\space}2\HL{\space}3]
1×3 Array{Int64,2}:
 1  2  3
\end{semiverbatim}
    \end{minipage}
    \\
    \begin{minipage}{1.0\linewidth}
      \vspace{2ex}
      \raggedright
      2D array (matrix) entry requires that rows are separated by semicolons:
      \vspace{-1ex}
\begin{semiverbatim}
\julia m = [1 2\HL{;} 3 4]
2×2 Array{Int64,2}:
 1  2
 3  4
\end{semiverbatim}
    \end{minipage}
    &
    \begin{minipage}{1.0\linewidth}
      \raggedright
      No literal entry syntax is there for higher-dimensional arrays.

      \+
      The \texttt{reshape()} function can be used to turn a vector
      into an $N$-dimensional array.
    \end{minipage}
  \end{tabular}
\end{frame}


\begin{frame}[fragile]
  \frametitle{Constructing arrays, I}
    \small
  Julia provides a few functions to provide commonly-used arrays.
  \begin{columns}
    \begin{column}{0.5\textwidth}
      \raggedright

      \begin{describe}{\texttt{zeros(type, size)}}
        A newly-created array filled with \texttt{0}'s.
\begin{lstlisting}
~\julia~ zeros((2,3))
2~$\times$~3 Array{Float64,2}:
 0.0  0.0  0.0
 0.0  0.0  0.0
\end{lstlisting}
      \end{describe}

      \begin{describe}{\texttt{trues(size)}}
        Same, but with boolean \texttt{true} entries.
      \end{describe}
    \end{column}
    \begin{column}{0.5\textwidth}
      \raggedright

      \begin{describe}{\texttt{ones(type, size)}}
        A newly-created array filled with \texttt{1}'s.
\begin{lstlisting}
~\julia~ ones(2)
2-element Array{Float64,1}:
 1.0
 1.0
\end{lstlisting}
      \end{describe}

      \begin{describe}{\texttt{falses(size)}}
        Same, but with boolean \texttt{false} entries.
      \end{describe}
    \end{column}
  \end{columns}

  \+
  Argument \texttt{type} can be omitted and defaults to \texttt{Float64}.
\end{frame}


\begin{frame}[fragile]
  \frametitle{Constructing arrays, II}
  \begin{columns}
    \small
    \begin{column}{0.5\textwidth}
      \raggedright

      \begin{describe}{\texttt{rand(type, size)}, \texttt{randn(type, size)}}
        A newly-created array filled with random numbers.

        \+
        \texttt{rand} picks entries using a uniform distribution,
        \texttt{randn} uses a normal one with $\mu=0$ and $\sigma=1$.
      \end{describe}

      \+
      \begin{describe}{\texttt{similar(a)}}
        Create an uninitialized array with the same element type and
        size as the given array \texttt{a}.
      \end{describe}
    \end{column}
    \begin{column}{0.5\textwidth}
      \raggedright

      \begin{describe}{\texttt{rand(S, size)}}
        A newly-created array filled with entries picked uniformly at
        random from collection \texttt{S}.
\begin{lstlisting}
~\julia~ rand("julia", (2,3))
2~$\times$~3 Array{Char,2}:
 'j'  'a'  'i'
 'l'  'u'  'a'
\end{lstlisting}
      \end{describe}
    \end{column}
  \end{columns}
\end{frame}


\begin{frame}[fragile]
  \frametitle{Accessing elements in vectors}
  You can access individual items in a vector using the postfix
  \texttt{[]} operator.

  \+
  Element indices start at 1.

\begin{lstlisting}
~\julia~ v[1]
1
~\julia~ v[2] * v[3]
6
\end{lstlisting}

  \+
  The keyword \texttt{end} can be used to get the last item of a vector:
\begin{lstlisting}
~\julia~ v[end]
5
\end{lstlisting}
\end{frame}


\begin{frame}[fragile]
  \frametitle{Slices}
  The notation \texttt{[$k$:$l$]} is used for accessing a \emph{slice}
  of an array (the items at positions $k$, $k+1$, \ldots, $l$).
\begin{lstlisting}
~\julia~ v[2:4]
3-element Array{Int64,1}:
 2
 3
 4
\end{lstlisting}

  \+ Again, one can use \texttt{end} in place of the number $l$ to
  mean the length of the array.

  \+ A single colon `\texttt{:}' abbreviates `\texttt{1:end}'.
\end{frame}


\begin{frame}[fragile]
  \frametitle{Accessing elements in arrays}
  \smaller
  For matrices and higher-dimensional arrays,
  indices must be separated by a comma:
\begin{lstlisting}
~\julia~ m[1,1]
1
~\julia~ m[end, end]  # last item in each dim
4

~\julia~ a = reshape(1:27, (3,3,3)); # 3-D array
~\julia~ a[1,2,3]
22
\end{lstlisting}

  Slicing is possible in each dimension separately;
  note that dimensions with just 1 index are dropped:
\begin{lstlisting}
~\julia~ a[:, 1:2, 1]  # 3x2(x1) slice
3~$\times$~2 Array{Int64,2}:
 1  4
 2  5
 3  6
\end{lstlisting}
\end{frame}


\begin{frame}[fragile]
  \frametitle{Dimensions of an array}
  \smaller

  \begin{columns}
    \begin{column}{0.5\textwidth}
      \raggedright

  \begin{describe}{\texttt{ndims(a)}}
    Return number of dimensions of an array.
\begin{semiverbatim}
\julia ndims(v)
1
\julia ndims(a)
3
\end{semiverbatim}
  \end{describe}

  \begin{describe}{\texttt{length(a)}}
    Return the number of elements in an array:
\begin{semiverbatim}
\julia length(v)
5
\julia length(a)
27
\end{semiverbatim}
  \end{describe}
    \end{column}
    \begin{column}{0.5\textwidth}
      \raggedright

  \begin{describe}{\texttt{size(a)}}
    Return tuple of maximum index in each dimension.
\begin{semiverbatim}
\julia size(v)
(3,)
\julia size(a)
(3, 3, 3)
\end{semiverbatim}

    Since indexing starts at 1, valid indices into \lstinline|i|-th
    array dimension range from 1 up to \lstinline|size(a,i)|
    \emph{(inclusive!)}.
  \end{describe}
    \end{column}
  \end{columns}
\end{frame}


\begin{frame}[fragile]
  \frametitle{Item mutation}
  You can replace items in an array by assigning them
  a new value:
\begin{semiverbatim}
\julia \HL{v[2] = 9};

\julia v
5-element Array\{Int64,1\}:
 \,\,1
 \HL{9}
 \,\,3
 \,\,4
 \,\,5
\end{semiverbatim}
\end{frame}

\begin{frame}[fragile]
  \frametitle{Slice mutation, I}

You can also replace an entire slice of an array:
\begin{semiverbatim}
\julia \HL{v[2:4] = [5,6,7]};
\julia v
5-element Array\{Int64,1\}:
 \,\,1
 \HL{5}
 \HL{6}
 \HL{7}
 \,\,5
\end{semiverbatim}

The new slice \emph{must} have the same length of the one it is replacing:
\begin{lstlisting}
~\julia~ v[2:4] = [5,6,7,8,9];
~\ERROR{DimensionMismatch("tried to assign 5 elements to 3 destinations")}~
\end{lstlisting}
\end{frame}


\part{Loops}
\begin{frame}[fragile]
  \frametitle{\texttt{for}-loops}
    With the  \texttt{for} statement, you can \textbf{loop over the items of
    a range or collection}:
\begin{lstlisting}
for i in 1:3
  # loop block
  println(i*i)
end
\end{lstlisting}

  \+
  To break out of a \texttt{for} loop, use the \texttt{break}
  statement.

  \+
  To jump to the next iteration of a \texttt{for} loop, use the
  \texttt{continue} statement.
\end{frame}


\begin{frame}[fragile]
  The \texttt{for} statement can be used \\ to loop over elements in \emph{any collection}.

  \+
  \begin{columns}[c]
    \smaller
    \begin{column}{0.5\textwidth}
      \raggedright

      Loop over vectors:
      \vspace{-1ex}
\begin{lstlisting}
~\julia~
  for n in [1,2,3]
    println(n)
  end
1
2
3
\end{lstlisting}

      \+ Loop over strings:
      \vspace{-1ex}
\begin{lstlisting}
~\julia~
  for ch in "abc"
    println(ch)
  end
a
b
c
\end{lstlisting}
    \end{column}
    \begin{column}{0.5\textwidth}
      \raggedright

      Loop over matrices and arrays:
      \vspace{-1ex}
\begin{lstlisting}
~\julia~
  for n in [1 2; 3 4]
    println(n)
  end
1
3
2
4
\end{lstlisting}

      \+ Loop over tuples:
      \vspace{-1ex}
\begin{lstlisting}
~\julia~ for x in (4, 2)
         println(x)
       end
4
2
\end{lstlisting}

    \end{column}
  \end{columns}
\end{frame}


\begin{frame}[fragile]
  \begin{exercise*}[3.A]
    Write a function \texttt{avg(v)} that takes a vector \texttt{v} of
    numbers and returns their mean value.
  \end{exercise*}

  \+
  \begin{exercise*}[3.B]
    Write a function \texttt{tr(m)} which, given a (square) matrix
    \texttt{m}, returns its \emph{trace}, i.e., sum of the elements
    along the main diagonal.
  \end{exercise*}

  \+
  \begin{exercise*}[3.C]
    Write a function \texttt{eye($n$)} which, given an integer
    $n>0$, returns the $n\times n$ identity matrix.
  \end{exercise*}

  \+
  \begin{exercise*}[3.D]
    Write a function \texttt{flip(m)} which, given a matrix
    \texttt{m}, returns its transpose.
  \end{exercise*}
\end{frame}


\part{Vector operations}
\begin{frame}
  \frametitle{Vector operations}
  By virtue of having a 1D index set, Vectors support a few operations
  that generic array cannot.

  \+
  They are quickly recalled here.
\end{frame}


\begin{frame}[fragile]
  \frametitle{Slice mutation, II}

  Function \texttt{splice!} can be used to replace a slice of a \emph{vector}
  with one of a different length:
\begin{semiverbatim}
\julia \HL{splice!(v, 2:4, [5,6,7,8,9])};
\julia v
7-element Array\{Int64,1\}:
 \,\,1
 \HL{5}
 \HL{6}
 \HL{7}
 \HL{8}
 \HL{9}
 \,\,5
\end{semiverbatim}
\end{frame}


\begin{frame}
  \frametitle{The meaning of the bang!}

  By convention, a ``bang'' character `\texttt{!}' marks functions that \emph{mutate}
  their first argument \emph{in-place}.

  \+ In other words, the bang signals that something is going to be
  changed destructively without an easy way to undo the effects.

  \+
  This convention is followed consistently in Julia's standard library.
\end{frame}


\begin{frame}
  \frametitle{Operating on vectors, I}
  \smaller

  \begin{describe}{\ttfamily push!(v,e), pushfirst!(v,e)}
    Extend vector \texttt{v} and insert item \texttt{e} as last
    (\texttt{push!}) or first (\texttt{pushfirst!}).
  \end{describe}

  \begin{describe}{\ttfamily pop!(v), popfirst!(v)}
    Remove the last (\texttt{pop!}) or first (\texttt{popfirst!}) item
    in vector \texttt{v} and return it.
  \end{describe}

  \begin{describe}{\ttfamily append!(v1,v2), prepend!(v1,v2)}
    Extend \texttt{v1} by adding elements of \texttt{v2} to its start or end.
  \end{describe}

  \begin{describe}{\ttfamily insert!(v, index, item)}
    Insert an item into \texttt{v} at the given index.
  \end{describe}

  \begin{describe}{\ttfamily deleteat!(v, index)}
    Remove the item at the given index and return the modified vector~\texttt{v}.
    Subsequent items are shifted to fill the resulting gap.
  \end{describe}

  \+
  \begin{references}
    \begin{itemize}
    \item \url{https://docs.julialang.org/en/v1/base/arrays/}
  \end{itemize}
  \end{references}
\end{frame}

\begin{frame}
  \frametitle{Operating on vectors, II}

% resize!(a::Vector, n::Integer) -> Vector
% Resize a to contain n elements. If n is smaller than the current collection length, the first n elements will be retained. If n is larger, the new elements are not guaranteed to be initialized.

\begin{describe}{\ttfamily sort(v), sort!(v)}
  Return a sorted copy (\texttt{sort}) of vector \texttt{v}, or sort
  it in-place \texttt{(sort!)}.  Use \texttt{sort(v, rev=true)} to
  reverse the ordering.
\end{describe}

  \begin{describe}{\ttfamily reverse(v), reverse!(v)}
    Return a copy of \texttt{v} reversed from start to end;
    or reverse it in-place.
  \end{describe}

% sortperm(v; alg::Algorithm=DEFAULT_UNSTABLE, lt=isless, by=identity, rev::Bool=false, order::Ordering=Forward)
% Return a permutation vector I that puts v[I] in sorted order.

  \+
  \begin{references}
    \begin{itemize}
    \item \url{https://docs.julialang.org/en/v1/base/sort/}
  \end{itemize}
  \end{references}
\end{frame}


\begin{frame}
  \begin{exercise*}[3.E]
    Write a function \texttt{median(v)} that takes a vector \texttt{v}
    of numbers and returns the median value.
  \end{exercise*}

  \+
  \begin{exercise*}[3.F]
    Write a function \texttt{evens(v)} that takes a list \texttt{v} of
    integers and returns a list of all the even ones.
  \end{exercise*}

  \+
  \begin{exercise*}[3.G]
    Write a function \texttt{deviation(v, u)} that takes a vector \texttt{v} of
    numbers and a single value \texttt{m} returns a list with the difference of
    \texttt{u} and each element \texttt{x} of \texttt{v}.
  \end{exercise*}
\end{frame}

%%%% MAP/REDUCE, ANONYMOUS FN, ARRAY COMPREHENSIONS?

\begin{frame}[fragile]
  \frametitle{map, reduce, filter (1)}

  Constructing a new collection by looping over a given one and applying a function
  on all elements is so common that there are specialized functions for that:

  \begin{describe}{\texttt{map(fn, a)}}
    Return a new collection formed by applying function \texttt{fn(x)}
    to every element \texttt{x} of array \texttt{a}.
  \end{describe}

  \+
  \begin{describe}{\texttt{filter(fn, a)}}
    Return a new collection formed by elements \texttt{x} of array
    \texttt{a} for which \texttt{fn(x)} evaluates to a \texttt{true} value.
  \end{describe}

  \+
  \begin{seealso}
    \url{https://en.wikibooks.org/wiki/Introducing_Julia/Functions#Map}
  \end{seealso}
\end{frame}


\begin{frame}[fragile]
  \frametitle{map, reduce, filter (2)}

  \begin{describe}{reduce(fn2, a)}
    Apply \emph{associative} 2-ary function \texttt{fn2(x,y)} to the
    first two items \texttt{x} and \texttt{y} of collection \texttt{a}, then
    apply \texttt{fn2} to the result and the third element of
    \texttt{L}, and so on until all elements have been processed ---
    return the final result.
  \end{describe}

  \+
  \begin{seealso}
    \url{https://en.wikibooks.org/wiki/Introducing_Julia/Functions#Reduce_and_folding}
  \end{seealso}
\end{frame}


\begin{frame}[fragile]
  This is how you could rewrite Exercises~3.F and~3.G using \texttt{map} and
  \texttt{filter}.

  \begin{columns}
    \begin{column}{0.5\linewidth}
\begin{jl}
# *** Exercise 3.F ***
function evens(v)
  filter(iseven, v)
end
\end{jl}
\end{column}
\begin{column}{0.5\linewidth}
\begin{jl}
# *** Exercise 3.G ***
function deviation(v, u)
  # note: can define
  # func's in func's!
  delta(x) = abs(x-u)
  return map(delta, v)
end
\end{jl}
\end{column}
\end{columns}
\end{frame}


\begin{frame}[fragile,fragile]
  \frametitle{Anonymous functions}
  \small
  With higher-order functions, it is sometimes convenient being able
  to use functions defined ``on the spot''.

  \+ Julia allows an alternate syntax for functions, specifying names
  of parameters and an expression to compute:
\begin{semiverbatim}
\julia x -> if x>0; exp(-(1/x)^2); else 0; end
\#21 (generic function with 1 method)
\end{semiverbatim}

  \+ Note that functions defined this way are not assigned a name!
  This syntax is meant to produce a function to be immediately used:
\begin{semiverbatim}
\julia map((x -> iseven(x) ? "odd" : "even"),
           rand(Int64, 3))
3-element Array{String,1}:
 "odd"
 "even"
 "odd"
\end{semiverbatim}
\end{frame}


\begin{frame}[fragile]
  \frametitle{The `do' notation, I}
  % Anonymous functions are convenient for short expressions;
  % for longer and more elaborate computations,
  % \texttt{map()}, \texttt{filter()}, \texttt{reduce()}
  % support yet another way of specifying a computation to be applied:
  \begin{columns}
    \begin{column}{0.5\textwidth}
\begin{jl}
# turn array of int's
# into corresponding
# Roman numerals
~\HL{map(a)}~ do num
  if num == 1
    return "I"
  elseif num == 2
    return "II"
    # ...
  end
end # close `do` block
\end{jl}
    \end{column}
    \begin{column}{0.5\textwidth}
      \raggedright

      Any function whose \emph{first} argument is a function can be
      used with \texttt{do}: e.g., \texttt{map()}, \texttt{filter()},
      etc.

      \+ The expression defined after \texttt{do} is inserted into the
      function call as the first argument.
    \end{column}
  \end{columns}
\end{frame}


\begin{frame}[fragile]
  \frametitle{The `do' notation, II}
  \begin{columns}
    \begin{column}{0.5\textwidth}
\begin{jl}
# turn array of int's
# into corresponding
# Roman numerals
map(a) ~\HL{\textbf{do}}~ num
  if num == 1
    return "I"
  elseif num == 2
    return "II"
    # ...
  end
~\HL{\textbf{end}}~ # close `do` block
\end{jl}
    \end{column}
    \begin{column}{0.5\textwidth}
      \raggedright

      The \lstinline|do| block must be closed by a corresponding
      \lstinline|end|.
    \end{column}
  \end{columns}
\end{frame}

\begin{frame}[fragile]
  \frametitle{The `do' notation, III}
  \begin{columns}
    \begin{column}{0.5\textwidth}
\begin{jl}
# turn array of int's
# into corresponding
# Roman numerals
map(a) do ~\HL{num}~
  if num == 1
    return "I"
  elseif num == 2
    return "II"
    # ...
  end
end # close `do` block
\end{jl}
    \end{column}
    \begin{column}{0.5\textwidth}
      \raggedright

      After \lstinline|do| there comes a parameter list:
      e.g., `\lstinline|do x,y|' would substitute \texttt{x} and
      \texttt{y} in the expression following.
    \end{column}
  \end{columns}
\end{frame}


\part{Type conversion}
\begin{frame}[fragile]
  \frametitle{Type conversion on assignment, I}
  When assigning a value to an array item or slice, automatic
  conversion to the element type of the array is attempted:
\begin{lstlisting}
~\julia~ ~\HL{v[3] = 5.0}~;  # converted to Int64
~\julia~ v
4-element Array{Int64,1}:
 ~\,\,~1
 ~\,\,~3
 ~\HL{5}~
 ~\,\,~7
\end{lstlisting}
\end{frame}


\begin{frame}[fragile,fragile]
  \frametitle{Type conversion on assignment, II}
  When assigning a value to an array item or slice, automatic
  conversion to the element type of the array is \emph{attempted} ---
  and it may fail!
\begin{lstlisting}
~\julia~ v[3] = 5.5
~\ERROR{InexactError: Int64(5.5)}~

~\julia~ v[3] = "five"
~\ERROR{MethodError: Cannot `convert` an object of type String to an object of type Int64}~

\end{lstlisting}
\end{frame}


\begin{frame}[fragile,fragile]
  \frametitle{Choosing the type of array elements, I}
  Julia guesses the type of elements based on the initial set of items:
\begin{lstlisting}
~\julia~ typeof([1,2,3])
Array{Int64,1}
~\julia~ typeof([1.0, 2.0, 3.0])
Array{Float64,1}
\end{lstlisting}

  \+ Arrays that can hold any value are possible, using the
  \texttt{Any} element type:
\begin{lstlisting}
~\julia~ typeof([1, "two", 3.0])
Array{Any,1}
\end{lstlisting}
\end{frame}


\begin{frame}[fragile,fragile]
  \frametitle{Choosing the type of array elements, II}
  You can convert any array to one that has a different element type,
  like you convert any Julia value to another type, using
  \texttt{convert} or the type constructor:
\begin{lstlisting}
~\julia~ b = Array{Float64,1}([1,2,3])
3-element Array{Float64,1}:
 1.0
 2.0
 3.0

~\julia~ c = convert(Array{Float64,1}, [1, 2, 3])
3-element Array{Float64,1}:
 1.0
 2.0
 3.0
\end{lstlisting}
\end{frame}


\begin{frame}[fragile,fragile]
  \frametitle{Choosing the type of array elements, III}

  Still, individual elements must be convertible!
\begin{lstlisting}
~\julia~ convert(Array{Float64,1}, [1, "two", 3])
~\ERROR{MethodError: Cannot `convert` an object of type String to an object of type Float64}~

\end{lstlisting}
\end{frame}
\end{document}

%%% Local Variables:
%%% mode: latex
%%% TeX-master: t
%%% End:
