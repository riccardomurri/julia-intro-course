\documentclass[english,serif,mathserif,xcolor=pdftex,dvipsnames,table]{beamer}

\usepackage[T1]{fontenc}
\usepackage[utf8x]{inputenc}

\usetheme[informal]{s3it}
\usepackage{s3it}

\title[Introduction to Julia]{%
  A Short and Incomplete Introduction to Julia
}
\subtitle{\bfseries Part 7: Ending remarks}
\author[R.~Murri]{%
  \textbf{Riccardo Murri} \texttt{<riccardo.murri@uzh.ch>}
  \\
  S3IT: Services and Support for Science IT,
  \\
  University of Zurich
}
\date{September 12, 2019}


\begin{document}

% title frame
\maketitle


\begin{frame}
  \frametitle{There is more to Julia than this\ldots}

  The time and scope of this course is quite limited.

  \+
  Here is an (incomplete) list of Julia features that you might
  want to look up as you become more experienced in the language:
  % FIXME: mettere un link per ciascuna di queste!
  \begin{itemize}
  \item Dicts, Sets, and other data structures
  \item Types and Multiple dispatch
  \item Interfaces
  \item Distributed and parallel programming
  \item Metaprogramming and macros
  \item \ldots and a thriving ecosystem of libraries!
  \end{itemize}
\end{frame}


% \begin{frame}
%   \frametitle{Other useful libraries}
% \end{frame}


\begin{frame}[fragile]
  \frametitle{Where to go now?}

  \begin{itemize}
  \item \textbf{The Julia Express},
    {\small \url{http://bogumilkaminski.pl/files/julia_express.pdf}}
  \item \textbf{Think Julia},
    {\small\url{https://benlauwens.github.io/ThinkJulia.jl/latest/book.html}}
  \item {An Introduction to DataFrames},
    {\small\url{https://github.com/bkamins/Julia-DataFrames-Tutorial}}
  \item {Parallel Computing course on Julia Academy},
    {\small\url{https://juliaacademy.com/p/parallel-computing}}
  \end{itemize}
\end{frame}


\begin{frame}
  \frametitle{Material from this course}
  \small
  All the material from this course is online:
  \url{http://github.com/riccardomurri/julia-intro-course/}

  \+
  To run the notebooks and all the code from the course:
  \begin{itemize}
  \item Install \href{https://julialang.org/downloads/}{Julia}
    on your computer; supports Linux,
    MacOSX, Windows: \url{https://julialang.org/downloads/}
  \item Use
    \href{http://gc3-uzh-ch.github.io/elasticluster}{ElastiCluster} to
    build dedicated servers and clusters on the cloud:
    \url{http://gc3-uzh-ch.github.io/elasticluster}
  \end{itemize}
\end{frame}


% \begin{frame}
%   \begin{quote}
%     ``Reverend fathers, my letters were not wont either to be so prolix, or to
%     follow so closely on one another. Want of time must plead my excuse for both
%     of these faults. The present letter is a very long one, simply because I had
%     no leisure to make it shorter.''
%   \end{quote}

%   \+
%   {\small\em
%     Blaise Pascal, The Provincial Letters,
%     \href{https://ebooks.adelaide.edu.au/p/pascal/blaise/p27pr/part17.html}{Letter XVI}}
% \end{frame}

\begin{frame}
  \begin{quote}
    ``Mes Révérends Pères, mes Lettres n'avaient pas accoutumé de se suivre de
    si près, ni d'être si étendues. Le peu de temps que j'ai eu a été cause de
    l'un et de l'autre. Je n'ai fait celle-ci plus longue que parce que je n'ai
    pas eu le loisir de la faire plus courte.''
  \end{quote}

  \+
  \begin{flushright}
    \small\em
    --- Blaise Pascal, \\
    \href{https://www.ebooksgratuits.com/ebooksfrance/pascal_les_provinciales.pdf}{Seizième
      lettre aux révérends pères jésuites}
  \end{flushright}
\end{frame}


\begin{frame}
  \begin{center}
    {\Huge\bfseries Thanks!}
    \\[2em]
    {\large\itshape\color{gray} Now go write some Julia ;-)}
  \end{center}
\end{frame}

\end{document}

%%% Local Variables:
%%% mode: latex
%%% TeX-master: t
%%% End:
